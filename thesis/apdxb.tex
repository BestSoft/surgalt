\chapter{Юунаас эхлэх вэ?}
\label{apdx:B}
Мэдээж \LaTeX{} системийг өөрийн компьютерт суулгах хэрэгтэй. Үүний тулд:
\begin{itemize}
	\item \LaTeX татан авч суулгах. MikTex, TeX Live гээд олон янзын хувилбар бий. Windows системд хамгийн тохиромжтой нь MikTex: \url{http://www.miktex.org}
	\item \LaTeX{} -тай ажиллах орчинг татан авч суулгах. Ер нь дурын редактор болно. Гэхдээ зориулалтынх байвал илүү тохиромжтой(\url{http://en.wikipedia.org/wiki/Comparison_of_TeX_editors}). Үнэтэй системээс WinEdt: \url{http://www.winedt.com}, Үнэгүйгээс TeXstudio: \url{http://texstudio.sourceforge.net} -г илүүд үздэг.
	\item Монголоор алдаагүй бичихэд ''Open Office'' толь: \url{http://wiki.openoffice.org/wiki/Dictionaries} хэрэг болно.
	\item \LaTeX{} -ийн гарын авлага (интернетээр дүүрэн). Эхэлж үзэх сонгодог материал бол ''The Not So Short Introduction to \LaTeXe. Маш олон хэл дээр хөрвүүлэгдсэн. Түүний дотор Монгол: \url{http://www.ctan.org/tex-archive/info/lshort}. Мөн Wikibook: \url{http://en.wikibooks.org/wiki/LaTeX}
	\item \LaTeX{} сонирхогчдыг татдаг булан: \url{http://www.tug.org/begin.html}, сонирхолтой жишээ: \url{http://www.texample.net/tikz/}
	\item Бэлэн загвар: \url{http://www.latextemplates.com/}
	 
\end{itemize}

