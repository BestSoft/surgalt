\documentclass[a4paper,11pt]{report}
\usepackage[utf8]{inputenc} % UTF8 юникод текст оруулах
\usepackage[T2A]{fontenc} % кирил үсгийн кодчилол
\usepackage[mongolian]{babel} % олон хэлний текст

\usepackage{graphicx} % зураг оруулах
\usepackage{epstopdf} 
\usepackage{latexsym} 
\usepackage{rotating} % эргүүлэх
\usepackage{fancybox} % хүрээлэх
%\usepackage{amsmath} % гоё математикийн тэмдэг
\usepackage{bm} % дармал математик
\usepackage{color} % өнгөөр бичих
\usepackage{fancyhdr} % хуудасны хөл, толгой
%\usepackage[backend=biber, url=true, natbib=true]{biblatex}
\usepackage{hyperref}
\hypersetup{colorlinks=true}

%\bibliographystyle{plain}
%\bibliography{ref.bib}
%\addbibresource{ref.bib}
\bibliographystyle{abbrv}

\usepackage{pgf}
\usepackage{tikz} % зураг зурах
\usetikzlibrary{arrows,automata}

\usepackage{thesis} % ШУТИС -ийн тезисийн формат

\pagestyle{fancy}

\begin{document}

\pagenumbering{roman}

% Гарчиг, зохиогч, горилсон зэрэг, тэнхим/багийн нэр болон он сарыг оруулах.
% Оруулсны дараах байдал:
%\thesistitle
%	{Диссертаци/тезисийн нэр}
%	{Зохиогчийн нэр}
%	{Горилсон зэрэг}
%	{Тэнхим/багийн нэр}
%	{Огноо}
%\thesistitle командын тодорхойлолт thesis.sty -д байгаа бөгөөд
% өөр бусад тохиргоог энд хийж болно.

\thesistitle
	{Цахим сургалтын цогц систем}
	{\emph{Т.Золбоо\\t.zolboo@must.edu.mn}}
	{\emph{''Компьютерийн ухаан''} -аар доктор}
	{\emph{Програмчлалын технологийн профессорын баг}}
	{\emph{2013-09-29}}

\tableofcontents


\pagenumbering{arabic}
\chapter{Оршил}
\section{Төслийн ажлын зорилго}
\section{Системийн хүрээ хязгаар}
\section{Нэр томъёоны тайлбар}
\chapter{Судалгаа}
\section{Хэрэглэгчийн судалгаа}
\section{Байгууллагын дүрэм журам}
\section{Байгууллагын мэдээллийн систем}
\section{Ажлын урсгалын диаграм (Workflow diagram)}
\chapter{Төслийн хэсэг}
\section{Системийн шаардлага}
\subsection{Системийн хэрэглэгчид}
\subsection{Функциональ шаардлага}
\subsection{Функциональ бус шаардлага}
\subsection{Юзкейз диаграм (Use Case diagram)}
\subsection{Юзкейз тодорхойлолт (Use case-ийн extension)}
\section{Системийн шинжилгээ}
\subsection{Технологийн шаардлага}
\subsection{Техникийн шаардлага}
\subsection{ОХД (Entity Relationship diagram)}
\subsection{Класс диаграм (Class diagram)}
\section{Системийн зохиомж}
\subsection{Үйл ажиллагааны диаграм (Activity Diagram)}
\subsection{Дарааллын диаграм (Sequence diagram)}
\subsection{Төлөв шилжилтийн диаграм (State diagram)}
\subsection{Бусад диаграмууд}
\subsection{Тестийн зохиомж}
\chapter{Дүгнэлт}
\chapter{Ашигласан бүтээлийн жагсаалт}
\section{Ном зүй}
\section{Веб сайтууд}
\addcontentsline{toc}{chapter}{Ном зүй}
\bibliography{ref}

%\printbibliography


% Хэрэв хавсралт байхгүй бол дараах мөрүүдийг хасаж болно.
% Олон хавсралттай бол нэмэж болно.
\appendix
\chapter{Хавсралтын нэр}
\label{apdx:A}
Энэ бол Хавсралт~\ref{apdx:A}.

Таны бүтээл олон тооны хавсралттай байж болно
(\emph{ө.х.}, \texttt{apdxb.tex}, \texttt{apdxc.tex}, \emph{г.м.}).
Хэрэв ямар ч хавсралт байхгүй бол хавсралттай холбоотой мөрүүдийг \texttt{thesis.tex} -аас хасна. Мөн \texttt{Makefile}-аас файлын нэрсийг хасна.

\chapter{Юунаас эхлэх вэ?}
\label{apdx:B}
Мэдээж \LaTeX{} системийг өөрийн компьютерт суулгах хэрэгтэй. Үүний тулд:
\begin{itemize}
	\item \LaTeX татан авч суулгах. MikTex, TeX Live гээд олон янзын хувилбар бий. Windows системд хамгийн тохиромжтой нь MikTex: \url{http://www.miktex.org}
	\item \LaTeX{} -тай ажиллах орчинг татан авч суулгах. Ер нь дурын редактор болно. Гэхдээ зориулалтынх байвал илүү тохиромжтой(\url{http://en.wikipedia.org/wiki/Comparison_of_TeX_editors}). Үнэтэй системээс WinEdt: \url{http://www.winedt.com}, Үнэгүйгээс TeXstudio: \url{http://texstudio.sourceforge.net} -г илүүд үздэг.
	\item Монголоор алдаагүй бичихэд ''Open Office'' толь: \url{http://wiki.openoffice.org/wiki/Dictionaries} хэрэг болно.
	\item \LaTeX{} -ийн гарын авлага (интернетээр дүүрэн). Эхэлж үзэх сонгодог материал бол ''The Not So Short Introduction to \LaTeXe. Маш олон хэл дээр хөрвүүлэгдсэн. Түүний дотор Монгол: \url{http://www.ctan.org/tex-archive/info/lshort}. Мөн Wikibook: \url{http://en.wikibooks.org/wiki/LaTeX}
	\item \LaTeX{} сонирхогчдыг татдаг булан: \url{http://www.tug.org/begin.html}, сонирхолтой жишээ: \url{http://www.texample.net/tikz/}
	\item Бэлэн загвар: \url{http://www.latextemplates.com/}
	 
\end{itemize}



\end{document}
